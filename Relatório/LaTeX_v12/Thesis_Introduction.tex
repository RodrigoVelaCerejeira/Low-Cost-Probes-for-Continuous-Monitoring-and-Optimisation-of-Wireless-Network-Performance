%%%%%%%%%%%%%%%%%%%%%%%%%%%%%%%%%%%%%%%%%%%%%%%%%%%%%%%%%%%%%%%%%%%%%%%%
%                                                                      %
%     File: Thesis_Introduction.tex                                    %
%     Tex Master: Thesis.tex                                           %
%                                                                      %
%     Author: Andre C. Marta                                           %
%     Last modified :  4 Mar 2024                                      %
%                                                                      %
%%%%%%%%%%%%%%%%%%%%%%%%%%%%%%%%%%%%%%%%%%%%%%%%%%%%%%%%%%%%%%%%%%%%%%%%

\chapter{Introduction}
\label{chapter:introduction}

The increase in wireless communications whether because of mobile devices 
or because the Internet of Things (IoT) created new challenges. 
More and more it is important to make sure that connectivity is reliable,
secure and efficient.
However, the traditional methods of network monitorization are not affordable
and flexible enough to meet the new requirements.

In this context, the opportunity arises to explore new alternatives to
monitor and optimize wireless networks. Using Raspberry Pi devices, 
it is possible to create a low-cost and flexible solution at 
Instituto Superior T\'ecnico (IST) - Oeiras.


%%%%%%%%%%%%%%%%%%%%%%%%%%%%%%%%%%%%%%%%%%%%%%%%%%%%%%%%%%%%%%%%%%%%%%%%
\section{Motivation}
\label{section:motivation}

sss

%%%%%%%%%%%%%%%%%%%%%%%%%%%%%%%%%%%%%%%%%%%%%%%%%%%%%%%%%%%%%%%%%%%%%%%%
\section{Topic Overview}
\label{section:overview}

Provide an overview of the topic to be studied.


%%%%%%%%%%%%%%%%%%%%%%%%%%%%%%%%%%%%%%%%%%%%%%%%%%%%%%%%%%%%%%%%%%%%%%%%
\section{Objectives and Deliverables}
\label{section:objectives}

The main goal of this project besides developing new monotorization
tools and analysis is also the creation of a new website useful to visualize
the data collected by the devices. To store all this data, it is also necessary
to create a database that allows efficient storage and retrieval of the data.


%%%%%%%%%%%%%%%%%%%%%%%%%%%%%%%%%%%%%%%%%%%%%%%%%%%%%%%%%%%%%%%%%%%%%%%%
\section{Thesis Outline}
\label{section:outline}

Briefly explain the contents of each chapter.
