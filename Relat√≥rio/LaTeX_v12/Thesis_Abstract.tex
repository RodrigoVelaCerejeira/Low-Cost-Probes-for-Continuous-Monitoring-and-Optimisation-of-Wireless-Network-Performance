%%%%%%%%%%%%%%%%%%%%%%%%%%%%%%%%%%%%%%%%%%%%%%%%%%%%%%%%%%%%%%%%%%%%%%%%
%                                                                      %
%     File: Thesis_Abstract.tex                                        %
%     Tex Master: Thesis.tex                                           %
%                                                                      %
%     Author: Andre C. Marta                                           %
%     Last modified :  4 Mar 2024                                      %
%                                                                      %
%%%%%%%%%%%%%%%%%%%%%%%%%%%%%%%%%%%%%%%%%%%%%%%%%%%%%%%%%%%%%%%%%%%%%%%%

\section*{Abstract}

% Add entry in the table of contents as section
\addcontentsline{toc}{section}{Abstract}

Wireless communication technologies are evolving exponentially, with the 
introduction of new standards and devices, such as Internet of Things (IoT).
Is that why, day by day, it becomes more important to monitorize and optimize
the performance of wireless networks to ensure the quality of service betrothed.
This work pretends to develop a low-cost solution to monitorize continuasly the
performance of this netoworks, using Raspberry Pi devices.

The proposed solution consists in the creation of a system that collect and 
store fundamental performance metrics, such as latency, packet loss, donwload and
upload speed, and Round-Trip Time (RTT). This data are stored in a database 
(MariaDB) and, subsequently viewed on a website, allowing real-time analysis 
of network performance.

With this solution, besisdes the continuous monitorization, it is also easier
to identify failures and to analyze tendencies, allowing a better management and 
optimisation of the network. This system offers effective and low-cost tools to 
make sure the quality of the network, especially in budget-constrained scenarios.  


\vfill

\textbf{\Large Keywords:} keyword1, keyword2, keyword3,...

